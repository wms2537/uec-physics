% !TEX root = Formulas.tex  

\documentclass{article}
\usepackage{geometry}
 \geometry{
 a4paper,
 total={170mm,257mm},
 left=20mm,
 top=20mm,
 }
\usepackage[utf8]{inputenc}
\usepackage{empheq} % autoloads mathtols and amsmath

\title{Physics Formulas}
\author{Wei Meng Soh \thanks{Chong Hwa Independent High School Kuala Lumpur}}
\date{February 2021}

\begin{document}
\maketitle
\begin{abstract}
    This is a list of formulas for physics...
\end{abstract}

\section{Thermometry}
\subsection*{Type of thermometers}

\subsubsection*{Liquid thermometer}
Thermometric Property: \(\Delta V \propto \Delta \theta\)

\noindent Formulae:
\[\theta = \frac{\ell_\theta - \ell_0}{\ell_{100}-\ell_0} \times 100^{\circ} \mathrm{C} \quad, \quad T=\frac{\ell_T-\ell_{00}}{\ell_{tr}-\ell_{00}} \times 273.16\mathrm{~K} \]

\subsubsection*{Gas thermometer}
Thermometric Property: \(\Delta P \Delta V \propto \Delta \theta \quad (\mathrm{where} \ P = \rho gh)\)

\noindent Formulae:
\[\theta = \frac{P_\theta V_\theta - P_0V_0}{P_{100} V_{100} - P_0V_0} \times 100^{\circ} \mathrm{C} \quad, \quad T=\frac{P_T V_T}{P_{tr} V_{tr}} \times 273.16\mathrm{~K} \]

\subsubsection*{Resistance thermometer}
Thermometric Property: \(\Delta R \propto \Delta \theta \quad (\mathrm{where} \ (\mathrm{i}) R=\frac{P}{Q}\times S \ (\mathrm{ii})\ R_t=R_0(1+at+bt^2))\)

\noindent Formulae:
\[\theta = \frac{R_\theta - R_0}{R_{100}-R_0} \times 100^{\circ} \mathrm{C} \quad, \quad T=\frac{R_T}{R_{tr}} \times 273.16\mathrm{~K} \]


\subsubsection*{Thermoelectric thermometer}
Thermometric Property: \(\Delta \varepsilon \propto \Delta \theta\)

\noindent Formulae:
\[\theta = \frac{\varepsilon_\theta - \varepsilon_0}{\varepsilon_{100}-\varepsilon_0} \times 100^{\circ} \mathrm{C} \quad, \quad T=\frac{\varepsilon_T-\varepsilon_{00}}{\varepsilon_{tr}-\varepsilon_{00}} \times 273.16\mathrm{~K} \]

\section{Calorimetry }
\subsection*{Heat Capacity and specific heat capacity}

\subsubsection*{Heat Capacity}
\[C = \frac{Q}{\Delta T}\quad (\mathrm{JK^{-1}})\]

\subsubsection*{Specific Heat Capacity}
\[c = \frac{Q}{m\Delta T}\quad (\mathrm{Jkg^{-1}K^{-1}})\]

\subsubsection*{Molar Heat Capacity}
\[C_v = \frac{Q}{n\Delta T} \ (\mathrm{Jmol^{-1}K^{-1}})\quad , \quad C_p = \frac{Q}{n\Delta T} \ (\mathrm{Jmol^{-1}K^{-1}})\]

\subsection*{Measurement of specific heat capacity}

\subsubsection*{Method of Mixture}
\[mc(\theta_3-\theta_2)=m_wc_w(\theta_2-\theta_1)+m_cc_c(\theta_2-\theta_1)\]

\subsubsection*{Electrical Heating Method}
\[VIt=(mc_\ell+C)\Delta \theta\]

\subsubsection*{Continuous Flow Method (Callendar \& Barnes' method)}

\begin{empheq}[left=\empheqlbrace]{align}
    \ V_1I_1t = m_1c(\theta_2-\theta_1)+ht\\
    \ V_2I_2t = m_2c(\theta_2-\theta_1)+ht
\end{empheq}

\subsection*{Specific Latent Heat}
\[L_f = \frac{Q}{m}\quad (Jkg^{-1})\quad ,\quad L_v = \frac{Q}{m}\quad (Jkg^{-1})\]

\subsubsection*{Finding specific latent heat of fusion of ice}
\[m_1c_w(\theta_1-\theta_2)+C(\theta_1-\theta_2)=mL_f+mc_w(\theta_2-0)\]

\subsubsection*{Finding specific latent heat of vaporisation of water}
\[mL_v+mc_w(100-\theta_2)=(m_1c_w+C)(\theta_2-\theta_1)\]

\subsection*{Thermal Expansion of solid}

\subsubsection*{Linear Expansion}
\[\alpha=\frac{l_2-l_1}{(\theta_2-\theta_1)l_1}\quad \Rightarrow\quad l_2=l_1[1+\alpha(\theta_2-\theta_1)]\]

\subsubsection*{Area Expansion}
\[\beta=\frac{A_2-A_1}{(\theta_2-\theta_1)A_1}\quad \Rightarrow\quad A_2=A_1[1+\beta(\theta_2-\theta_1)]\]
\[\beta=2\alpha\]

\subsubsection*{Volume Expansion}
\[\gamma=\frac{V_2-V_1}{(\theta_2-\theta_1)V_1}\quad \Rightarrow\quad V_2=V_1[1+\gamma(\theta_2-\theta_1)]\]
\[\gamma=3\alpha\]

\subsection*{Thermal Expansion of Liquid}
\[\gamma_\ell=\frac{V_1-V_0}{V_0\Delta \theta}\quad \Rightarrow\quad V_1=V_0(1+\gamma_\ell \Delta \theta)\]
\[3\alpha_c=\gamma_c=\frac{V_1'-V_0}{V_0\Delta \theta}\quad \Rightarrow\quad V_1'=V_0(1+\gamma_c \Delta \theta)\]
\[\gamma_a=\frac{V_1-V_1'}{V_0\Delta \theta}\quad \Rightarrow\quad \gamma_\ell=\gamma_a+\gamma_c\]

\section{Transmission of Heat}
\subsection*{Conduction}

\subsubsection*{Temperature Gradient}
\[\frac{d\theta}{dx}=\frac{\theta_2-\theta_1}{\ell}\quad (\theta_2>\theta_1)\]
\[\frac{Q}{t}\quad \propto \quad \frac{\theta_2-\theta_1}{\ell}\quad (\theta_2>\theta_1)\]
\[\frac{Q}{t}\quad \propto \quad A\]
\[\Rightarrow\quad \frac{Q}{t}=kA\;\frac{\theta_2-\theta_1}{\ell}\quad (\theta_2>\theta_1)\]
\[\frac{dQ}{dt}=kA\;\frac{d\theta}{dx}\]

\subsubsection*{Heat flow through compound bar}
\[\left(\frac{Q}{t}\right)_1=\left(\frac{Q}{t}\right)_2=\left(\frac{Q}{t}\right)_3\]
\[k_1A\;\frac{\theta_1-\theta_2}{\ell_1}=k_2A\;\frac{\theta_2-\theta_3}{\ell_2}=k_3A\;\frac{\theta_3-\theta_4}{\ell_3}\quad (\theta_1>\theta_2>\theta_3>\theta_4)\]

\subsubsection*{Measuring thermal conductivity of good conductor}
\[Rate\ of\ heat\ flow = mc_w(\theta_4-\theta_3)\]
\[k=\frac{mc_w(\theta_4-\theta_3)}{A(\theta_2-\theta_1)}\times \ell\]

\subsubsection*{Thermal Resistance}
\[\frac{Q}{t}=\frac{\Delta \theta}{R_\theta}\]
\[R_\theta=\frac{\ell}{kA}\]

When in series, 
\[\mathrm{Total\ thermal\ resistance}=R_{\theta_1}+R_{\theta_2}\]
\[\frac{Q}{t}=\frac{\mathrm{temperature\ difference}}{\mathrm{total\ thermal\ resistance}}\]

\subsubsection*{Wein's displacement law}
\[\lambda \propto \frac{1}{T}\quad (\lambda \mathrm{\ is\ peak\ wavelength})\]
\[\lambda T=k \quad (k\mathrm{is\ Wein's\ constant,\ }2.93\times10^{-3}mK)\]

\subsubsection*{Stefan's law}
\[E \propto T^4\quad (E=\frac{Q}{At}\mathrm{,\ energy\ emitted\ per\ second\ per\ unit\ surface})\]
\[E=\sigma T^4 \quad (\sigma\mathrm{\ is\ Stefan's\ constant,\ }5.67\times10^{-8}Wm^{-2}K^{-4})\]


\section{Optics}
\subsection*{Reflection}

\[\frac{1}{u} + \frac{1}{v} = \frac{1}{f}\quad(\mathrm{where}\ u=\mathrm{object\ distance}, v=\mathrm{image\ distance}, f=\mathrm{focal\ length} )\]
\[\mathrm{Linear\ magnification:}\quad m=\frac{v}{u}=\frac{l}{h}=\frac{v}{f}-1\]
\[\mathrm{Angular\ magnification:}\quad m=\frac{\beta}{\alpha}=\frac{v}{f}-1\]
No. of images:
\[\quad n=\frac{360^\circ}{\theta}-1\quad(\mathrm{if\ n\ is\ even\ or\ n\ is\ odd\ when\ object\ lies\ on\ angle\ bisector})\]
\[\quad n=\frac{360^\circ}{\theta}\quad(\mathrm{if\ n\ is\ odd\ when\ object\ does\ not\ lie\ on\ angle\ bisector})\]

\subsection*{Refraction}

Snell's law: \(n_1\sin i_1=n_2\sin i_2\)
\[if\ i=c\ and\ r=90^\circ,\ \frac{n}{n_a}=\frac{1}{\sin c}\]
\[\frac{n_2}{n_1}=\frac{\sin i}{\sin r}=\frac{v_1}{v_2}=\frac{\lambda_1}{\lambda_2}=\frac{t}{t'}\quad(\mathrm{light\ passes\ from\ 1\ to\ 2})\]

Difference between real and apparent depth:
\[d=t(1-\frac{1}{n})\]

Refraction at prism:
\[\mathrm{Angle\ of\ deviation:\ }d=i_1+i_2-A\ mathrm{where}\ A=r_1+r_2\]
\[\mathrm{Minimum\ Angle\ of\ deviation:\ }d=2i-A,\ (i_1=i_2=i,\ r_1=r_2=r,\ A=2r)\ or\ n=\frac{\sin\frac{A+D}{2}}{\sin\frac{A}{2}}\]
\[\mathrm{Maximum\ Angle\ of\ deviation:\ }i_1=90^\circ\ or\ i_2=90^\circ\]

Power of a lens:
\[P=\frac{1}{f},\ f\ \mathrm{is\ in\ metres}\]

Lensmaker's equation:
\[\frac{1}{f}=(\frac{n_2}{n_1}-1)(\frac{1}{r_1}+\frac{1}{r_2}),\ (r\ \mathrm{is\ +\ when\ convex\ towards\ optically\ less\ dense\ medium})\]

Combination of thin lenses:
\[\frac{1}{f}=\frac{1}{f_1}+\frac{1}{f_2}\]

Conjugate points:
\[xx'=f^2\]

Convex lens with fixed object and image:
\[f=\frac{s^2-l^2}{4d},\ (\mathrm{where}\ d=u+v\ \mathrm{and}\ l=\mathrm{distance\ between\ two\ convex\ lens)}\]
\[u+v>4f\ \Rightarrow\mathrm{2\ real\ images}\]
\[u+v=4f\ \Rightarrow\mathrm{1\ real\ image}\]
\[u+v<4f\ \Rightarrow\mathrm{no\ real\ images}\]

Magnifying glass: convex lens, \(u<f\), erect, magnified and virtual image
\[\mathrm{Normal\ adjustment:}\ v=D,\ M=\frac{\beta}{\alpha}=\frac{D}{f}-1\]
\[\mathrm{Abnormal\ adjustment:}\ v=\infty,\ M=\frac{D}{f}\ (D=-25cm)\]

Compound microscope: convex lens, \(f_o<f_e\), first image is magnified inverted real, second is magnified inverted virtual.
\[\mathrm{Normal\ adjustment:}\ v_e=D,\ M=\frac{\beta}{\alpha}=m_o\times m_e=(\frac{v_o}{f_o}-1)(\frac{-D}{f_e}-1)\]
\[\mathrm{Abnormal\ adjustment:}\ v_e=\infty,\ M=\frac{\beta}{\alpha}=m_o\times m_e=(\frac{v_o}{f_o}-1)(\frac{D}{f_e})\]

Astromomical telescope (Keplerian telescope): convex lenses with
\[f_o<f_e,\ f_o+f_e=d,\ u_o=\infty,\ v_o=f_o,\ u_e=f_e\] first image is diminished, inverted real, second is magnified inverted
\[\mathrm{Normal\ adjustment:}\ v_e=\infty,\ M=\frac{\beta}{\alpha}=m_o\times m_e=\frac{f_o}{f_e}\]
\[\mathrm{Abnormal\ adjustment:}\ v_e=D,\ M=\frac{\beta}{\alpha}=m_o\times m_e=(\frac{D+1}{D})(\frac{f_o}{f_e})\]



\end{document}