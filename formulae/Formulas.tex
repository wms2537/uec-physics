% !TEX root = Formulas.tex  

\documentclass{article}
\usepackage{geometry}
 \geometry{
 a4paper,
 total={170mm,257mm},
 left=20mm,
 top=20mm,
 }
\usepackage[utf8]{inputenc}
\usepackage{empheq} % autoloads mathtols and amsmath
\usepackage{pgfplots}
\pgfplotsset{width=10cm,compat=1.9}
\usepgfplotslibrary{external}
\tikzexternalize

\title{Physics Formulas}
\author{Wei Meng Soh \thanks{Chong Hwa Independent High School Kuala Lumpur}}
\date{February 2021}

\begin{document}
\maketitle
\begin{abstract}
    This is a list of formulas for physics...
\end{abstract}

\section{Thermometry}
\subsection*{Type of thermometers}

\subsubsection*{Liquid thermometer}
Thermometric Property: \(\Delta V \propto \Delta \theta\)

\noindent Formulae:
\[\theta = \frac{\ell_\theta - \ell_0}{\ell_{100}-\ell_0} \times 100^{\circ} \mathrm{C} \quad, \quad T=\frac{\ell_T-\ell_{00}}{\ell_{tr}-\ell_{00}} \times 273.16\mathrm{~K} \]

\subsubsection*{Gas thermometer}
Thermometric Property: \(\Delta P \Delta V \propto \Delta \theta \quad (\mathrm{where} \ P = \rho gh)\)

\noindent Formulae:
\[\theta = \frac{P_\theta V_\theta - P_0V_0}{P_{100} V_{100} - P_0V_0} \times 100^{\circ} \mathrm{C} \quad, \quad T=\frac{P_T V_T}{P_{tr} V_{tr}} \times 273.16\mathrm{~K} \]

\subsubsection*{Resistance thermometer}
Thermometric Property: \(\Delta R \propto \Delta \theta \quad (\mathrm{where} \ (\mathrm{i}) R=\frac{P}{Q}\times S \ (\mathrm{ii})\ R_t=R_0(1+at+bt^2))\)

\noindent Formulae:
\[\theta = \frac{R_\theta - R_0}{R_{100}-R_0} \times 100^{\circ} \mathrm{C} \quad, \quad T=\frac{R_T}{R_{tr}} \times 273.16\mathrm{~K} \]


\subsubsection*{Thermoelectric thermometer}
Thermometric Property: \(\Delta \varepsilon \propto \Delta \theta\)

\noindent Formulae:
\[\theta = \frac{\varepsilon_\theta - \varepsilon_0}{\varepsilon_{100}-\varepsilon_0} \times 100^{\circ} \mathrm{C} \quad, \quad T=\frac{\varepsilon_T-\varepsilon_{00}}{\varepsilon_{tr}-\varepsilon_{00}} \times 273.16\mathrm{~K} \]

\section{Calorimetry }
\subsection*{Heat Capacity and specific heat capacity}

\subsubsection*{Heat Capacity}
\[C = \frac{Q}{\Delta T}\quad (\mathrm{JK^{-1}})\]

\subsubsection*{Specific Heat Capacity}
\[c = \frac{Q}{m\Delta T}\quad (\mathrm{Jkg^{-1}K^{-1}})\]

\subsubsection*{Molar Heat Capacity}
\[C_v = \frac{Q}{n\Delta T} \ (\mathrm{Jmol^{-1}K^{-1}})\quad , \quad C_p = \frac{Q}{n\Delta T} \ (\mathrm{Jmol^{-1}K^{-1}})\]

\subsection*{Measurement of specific heat capacity}

\subsubsection*{Method of Mixture}
\[mc(\theta_3-\theta_2)=m_wc_w(\theta_2-\theta_1)+m_cc_c(\theta_2-\theta_1)\]

\subsubsection*{Electrical Heating Method}
\[VIt=(mc_\ell+C)\Delta \theta\]

\subsubsection*{Continuous Flow Method (Callendar \& Barnes' method)}

\begin{empheq}[left=\empheqlbrace]{align}
    \ V_1I_1t = m_1c(\theta_2-\theta_1)+ht\\
    \ V_2I_2t = m_2c(\theta_2-\theta_1)+ht
\end{empheq}

\subsection*{Specific Latent Heat}
\[L_f = \frac{Q}{m}\quad (Jkg^{-1})\quad ,\quad L_v = \frac{Q}{m}\quad (Jkg^{-1})\]

\subsubsection*{Finding specific latent heat of fusion of ice}
\[m_1c_w(\theta_1-\theta_2)+C(\theta_1-\theta_2)=mL_f+mc_w(\theta_2-0)\]

\subsubsection*{Finding specific latent heat of vaporisation of water}
\[mL_v+mc_w(100-\theta_2)=(m_1c_w+C)(\theta_2-\theta_1)\]

\subsection*{Thermal Expansion of solid}

\subsubsection*{Linear Expansion}
\[\alpha=\frac{l_2-l_1}{(\theta_2-\theta_1)l_1}\quad \Rightarrow\quad l_2=l_1[1+\alpha(\theta_2-\theta_1)]\]

\subsubsection*{Area Expansion}
\[\beta=\frac{A_2-A_1}{(\theta_2-\theta_1)A_1}\quad \Rightarrow\quad A_2=A_1[1+\beta(\theta_2-\theta_1)]\]
\[\beta=2\alpha\]

\subsubsection*{Volume Expansion}
\[\gamma=\frac{V_2-V_1}{(\theta_2-\theta_1)V_1}\quad \Rightarrow\quad V_2=V_1[1+\gamma(\theta_2-\theta_1)]\]
\[\gamma=3\alpha\]

\subsection*{Thermal Expansion of Liquid}
\[\gamma_\ell=\frac{V_1-V_0}{V_0\Delta \theta}\quad \Rightarrow\quad V_1=V_0(1+\gamma_\ell \Delta \theta)\]
\[3\alpha_c=\gamma_c=\frac{V_1'-V_0}{V_0\Delta \theta}\quad \Rightarrow\quad V_1'=V_0(1+\gamma_c \Delta \theta)\]
\[\gamma_a=\frac{V_1-V_1'}{V_0\Delta \theta}\quad \Rightarrow\quad \gamma_\ell=\gamma_a+\gamma_c\]

\section{Transmission of Heat}
\subsection*{Conduction}

\subsubsection*{Temperature Gradient}
\[\frac{d\theta}{dx}=\frac{\theta_2-\theta_1}{\ell}\quad (\theta_2>\theta_1)\]
\[\frac{Q}{t}\quad \propto \quad \frac{\theta_2-\theta_1}{\ell}\quad (\theta_2>\theta_1)\]
\[\frac{Q}{t}\quad \propto \quad A\]
\[\Rightarrow\quad \frac{Q}{t}=kA\;\frac{\theta_2-\theta_1}{\ell}\quad (\theta_2>\theta_1)\]
\[\frac{dQ}{dt}=kA\;\frac{d\theta}{dx}\]

\subsubsection*{Heat flow through compound bar}
\[\left(\frac{Q}{t}\right)_1=\left(\frac{Q}{t}\right)_2=\left(\frac{Q}{t}\right)_3\]
\[k_1A\;\frac{\theta_1-\theta_2}{\ell_1}=k_2A\;\frac{\theta_2-\theta_3}{\ell_2}=k_3A\;\frac{\theta_3-\theta_4}{\ell_3}\quad (\theta_1>\theta_2>\theta_3>\theta_4)\]

\subsubsection*{Measuring thermal conductivity of good conductor}
\[Rate\ of\ heat\ flow = mc_w(\theta_4-\theta_3)\]
\[k=\frac{mc_w(\theta_4-\theta_3)}{A(\theta_2-\theta_1)}\times \ell\]

\subsubsection*{Thermal Resistance}
\[\frac{Q}{t}=\frac{\Delta \theta}{R_\theta}\]
\[R_\theta=\frac{\ell}{kA}\]

When in series, 
\[\mathrm{Total\ thermal\ resistance}=R_{\theta_1}+R_{\theta_2}\]
\[\frac{Q}{t}=\frac{\mathrm{temperature\ difference}}{\mathrm{total\ thermal\ resistance}}\]

\subsubsection*{Wein's displacement law}
\[\lambda \propto \frac{1}{T}\quad (\lambda \mathrm{\ is\ peak\ wavelength})\]
\[\lambda T=k \quad (k\mathrm{is\ Wein's\ constant,\ }2.93\times10^{-3}mK)\]

\subsubsection*{Stefan's law}
\[E \propto T^4\quad (E=\frac{Q}{At}\mathrm{,\ energy\ emitted\ per\ second\ per\ unit\ surface})\]
\[E=\sigma T^4 \quad (\sigma\mathrm{\ is\ Stefan's\ constant,\ }5.67\times10^{-8}Wm^{-2}K^{-4})\]


\section{Optics}
\subsection*{Reflection}

\[\frac{1}{u} + \frac{1}{v} = \frac{1}{f}\quad(\mathrm{where}\ u=\mathrm{object\ distance}, v=\mathrm{image\ distance}, f=\mathrm{focal\ length} )\]
\[\mathrm{Linear\ magnification:}\quad m=\frac{v}{u}=\frac{l}{h}=\frac{v}{f}-1\]
\[\mathrm{Angular\ magnification:}\quad m=\frac{\beta}{\alpha}=\frac{v}{f}-1\]
No. of images:
\[\quad n=\frac{360^\circ}{\theta}-1\quad(\mathrm{if\ n\ is\ even\ or\ n\ is\ odd\ when\ object\ lies\ on\ angle\ bisector})\]
\[\quad n=\frac{360^\circ}{\theta}\quad(\mathrm{if\ n\ is\ odd\ when\ object\ does\ not\ lie\ on\ angle\ bisector})\]

\subsection*{Refraction}

Snell's law: \(n_1\sin i_1=n_2\sin i_2\)
\[if\ i=c\ and\ r=90^\circ,\ \frac{n}{n_a}=\frac{1}{\sin c}\]
\[\frac{n_2}{n_1}=\frac{\sin i}{\sin r}=\frac{v_1}{v_2}=\frac{\lambda_1}{\lambda_2}=\frac{t}{t'}\quad(\mathrm{light\ passes\ from\ 1\ to\ 2})\]

Difference between real and apparent depth:
\[d=t(1-\frac{1}{n})\]

Refraction at prism:
\[\mathrm{Angle\ of\ deviation:\ }d=i_1+i_2-A\ mathrm{where}\ A=r_1+r_2\]
\[\mathrm{Minimum\ Angle\ of\ deviation:\ }d=2i-A,\ (i_1=i_2=i,\ r_1=r_2=r,\ A=2r)\ or\ n=\frac{\sin\frac{A+D}{2}}{\sin\frac{A}{2}}\]
\[\mathrm{Maximum\ Angle\ of\ deviation:\ }i_1=90^\circ\ or\ i_2=90^\circ\]

Power of a lens:
\[P=\frac{1}{f},\ f\ \mathrm{is\ in\ metres}\]

Lensmaker's equation:
\[\frac{1}{f}=(\frac{n_2}{n_1}-1)(\frac{1}{r_1}+\frac{1}{r_2}),\ (r\ \mathrm{is\ +\ when\ convex\ towards\ optically\ less\ dense\ medium})\]

Combination of thin lenses:
\[\frac{1}{f}=\frac{1}{f_1}+\frac{1}{f_2}\]

Conjugate points:
\[xx'=f^2\]

Convex lens with fixed object and image:
\[f=\frac{s^2-l^2}{4d},\ (\mathrm{where}\ d=u+v\ \mathrm{and}\ l=\mathrm{distance\ between\ two\ convex\ lens)}\]
\[u+v>4f\ \Rightarrow\mathrm{2\ real\ images}\]
\[u+v=4f\ \Rightarrow\mathrm{1\ real\ image}\]
\[u+v<4f\ \Rightarrow\mathrm{no\ real\ images}\]

Magnifying glass: convex lens, \(u<f\), erect, magnified and virtual image
\[\mathrm{Normal\ adjustment:}\ v=D,\ M=\frac{\beta}{\alpha}=\frac{D}{f}-1\]
\[\mathrm{Abnormal\ adjustment:}\ v=\infty,\ M=\frac{D}{f}\ (D=-25cm)\]

Compound microscope: convex lens, \(f_o<f_e\), first image is magnified inverted real, second is magnified inverted virtual.
\[\mathrm{Normal\ adjustment:}\ v_e=D,\ M=\frac{\beta}{\alpha}=m_o\times m_e=(\frac{v_o}{f_o}-1)(\frac{-D}{f_e}-1)\]
\[\mathrm{Abnormal\ adjustment:}\ v_e=\infty,\ M=\frac{\beta}{\alpha}=m_o\times m_e=(\frac{v_o}{f_o}-1)(\frac{D}{f_e})\]

Astromomical telescope (Keplerian telescope): convex lenses with
\[f_o<f_e,\ f_o+f_e=d,\ u_o=\infty,\ v_o=f_o,\ u_e=f_e\] first image is diminished, inverted real, second is magnified inverted
\[\mathrm{Normal\ adjustment:}\ v_e=\infty,\ M=\frac{\beta}{\alpha}=m_o\times m_e=\frac{f_o}{f_e}\]
\[\mathrm{Abnormal\ adjustment:}\ v_e=D,\ M=\frac{\beta}{\alpha}=m_o\times m_e=(\frac{D+1}{D})(\frac{f_o}{f_e})\]

\section{Gases}
\subsection*{Gas Laws}

\subsubsection*{Boyle's Law}
\[P\propto\frac{1}{V}\quad(m,\ T\ \mathrm{is\ constant})\]
\[PV=constant\]
\[\Rightarrow P_1V_1=p_2V_2\]

\subsubsection*{Charles' Law}
\[V\propto T\quad(m,\ P\ \mathrm{is\ constant})\]
\[\frac{V}{T}=constant\]
\[\Rightarrow \frac{V_1}{T_1}=\frac{V_2}{T_2}\]

\subsubsection*{Pressure Law}
\[P\propto T\quad(m,\ P\ \mathrm{is\ constant})\]
\[\frac{P}{T}=constant\]
\[\Rightarrow \frac{P_1}{T_1}=\frac{P_2}{T_2}\]

\subsubsection*{Equation of State}
\[\frac{P_1V_1}{T_1}=\frac{P_2V_2}{T_2}=constant\]
\[\frac{PV}{T}=nR\quad(R=\mathrm{molar\ gas\ constant})\]
\[\Rightarrow PV=nRT=mR'T\quad(R'=\frac{R}{M})\]

\subsubsection*{Connected Gas Container}
\[n_1 + n_2 = n_1'+n_2'\]

\subsubsection*{Dalton's Law of Partial Pressure}
\[P_A=\frac{n_A}{V}RT\]
\[P_B=\frac{n_B}{V}RT\]
\[P_{total}=P_A+P_B\]

\subsection*{Kinetic Theory of Gas}
\[P=\frac{1}{3}\rho \overline{c^2}\quad(\overline{c^2}\ is\ mean\ squared\ speed)\]
\[\Rightarrow C_{rms}=\sqrt{\overline{c^2}}=\sqrt{\frac{3P}{\rho}}=\sqrt{\frac{3RT}{M}}\]

\subsubsection*{Temperature and Kinetic Energy}
Average translational kientic energy:
\[\frac{1}{2}m\overline{c^2}=\frac{3}{2}\frac{R}{N_A}T=\frac{3}{2}kT\quad(k\ \mathrm{is\ Boltzmann's\ constant})\]

\subsubsection*{Maxwell's Law of Equipartition of Energy}
\[\mathrm{For\ monoatomic\ gas,\ }f=3,\ \mathrm{average\ kinetic\ energy\ of\ one\ molecule}=\frac{3}{2}kT\]
\[\mathrm{For\ diatomic\ gas,\ }f=5,\ \mathrm{average\ kinetic\ energy\ of\ one\ molecule}=\frac{5}{2}kT\]
\[\mathrm{For\ polyatomic\ gas,\ }f=6,\ \mathrm{average\ kinetic\ energy\ of\ one\ molecule}=\frac{6}{2}kT=3kT\]

\subsubsection*{Internal Energy of a Gas, \(U\)}
\[U\propto T\]
\[U = N\times\mathrm{average\ E_k\ of\ one\ molecule}=\frac{f}{2}kT=\frac{f}{2}nRT\quad(Nk=nR)\]

\section{Laws of Thermodynamics}
\subsection*{First Law of Thermodynamics}

\[\Delta U=Q+W\quad(\Delta U=\mathrm{change\ in\ internal\ energy},\ Q=\mathrm{heat\ supplied\ to\ system},\ W=\mathrm{work\ done\ on\ gas})\]
\subsubsection*{Work done by gas}
\[W=\int_{V_1}^{V_2}-PdV\]

\subsubsection*{Molar Heat Capacity of Gas}
During constant volume:
\[\Delta U=Q_v=nC_{v,m}\Delta T\]
During constant pressure:
\[\Delta U=Q_p+W=nC_{p,m}\Delta T\]
Relationship:
\[Q_p>Q_v\]
\[C_p-C_v=R\]
\[\gamma=\frac{C_p}{C_v}=1+\frac{2}{f}\]

\subsubsection*{Isometric (Isochoric) Process}
\[\Delta V=0 \Rightarrow W=0\]
\[\Delta U=Q=nC_{v,m}\Delta T\]
\[\frac{P_1}{P_2}=\frac{T_1}{T_2}\]

\subsubsection*{Isobaric Process}
\[\Delta P=0 \Rightarrow W=-P(V_2-V_1)=-nR\Delta T\]
\[\Delta U=Q+W\Rightarrow nC_{v,m}\Delta T=nC_{p,m}\Delta T+(-P\Delta V)\]
\[\frac{V_1}{V_2}=\frac{T_1}{T_2}\]

\subsubsection*{Isothermal Process}
\[\Delta T=0 \Rightarrow \Delta U=0 \Rightarrow Q=-W\]
\[P_1V_1=P_2V_2\]
\[W=\int_{V_1}^{V_2}-PdV=-\int_{V_1}^{V_2}\frac{nRT}{V}dV=-nRT\ln\frac{V_2}{V_1}=-PV\ln\frac{V_2}{V_1}\]

\subsubsection*{Adiabatic Process}
\[\Delta Q=0 \Rightarrow \Delta U=W\]
\[TV^{\gamma -1}=constant \Rightarrow PV^\gamma=constant\]
\[W=\int_{V_1}^{V_2}-PdV=-PV^\gamma\int_{V_1}^{V_2}V^{-\gamma}dV=-\frac{P_2V_2-P_1V_1}{1-\gamma}\]

\subsubsection*{Isothermal vs Adiabatic}
Isothermal:
\[\frac{dP}{dV}=-\frac{P}{V}\]
Adiabatic:
\[\frac{dP}{dV}=-\gamma\frac{P}{V}\]
\[\left|{\frac{dP}{dV}}\right|_{adia}>\left|{\frac{dP}{dV}}\right|_{iso}\]

\section{Electrostatics}
\subsubsection*{Coulomb's Law}
\[\vec{F}_{1,2}=k\frac{q_1q_2}{r^2}\hat{r}_{1,2},\quad k=\frac{1}{4\pi\epsilon_0}\approx9\times10^9mF^{-1}\]

\subsubsection*{Electric Field Intensity}
\[\vec{E}=\frac{kQ}{r^2}\hat{r},\quad k=\frac{1}{4\pi\epsilon_0}\approx9\times10^9mF^{-1}\]

\subsubsection*{Electric Flux}
\[\phi=\oint_S\vec{E}\cdot\vec{dA}=\frac{\sum Q_{enc}}{\epsilon_0}\]

\subsubsection*{Electrostatic Potential Energy}
\[W_{ext}=\int_\infty^R\vec{F}_{ext}\cdot\vec{dr}=\int_R^\infty\vec{F}_{el}\cdot\vec{dr}=\frac{q_1q_2}{4\pi\epsilon_0}\int_R^\infty\frac{dr}{R^2}=\frac{q_1q_2}{4\pi\epsilon_0R}\]

\subsubsection*{Electric Potential}
Work done to bring unit charge from infinity to distance R from charge
\[V=\int_R^\infty\frac{\vec{F}_{el}}{q}\cdot\vec{dr}=\frac{Q}{4\pi\epsilon_0R}\]
\[W_{ext}=qV \Rightarrow W_{el}=-qV\]

\subsubsection*{Potential Difference}
\[V_A-V_B=\int_A^\infty\vec{E}\cdot\vec{dr}-\int_B^\infty\vec{E}\cdot\vec{dr}=\int_A^B\vec{E}\cdot\vec{dr}=\int_A^B\vec{E}\cdot\vec{d\ell}\quad(\mathrm{Electric\ force\ is\ conservative\ force})\]
Change in potential energy:
\[q(V_A-V_B)=K_B-K_A\]

\subsubsection*{Potential Gradient}
\[\vec{E}=\frac{Q}{4\pi\epsilon_0r^2}\hat{r},\quad V=\frac{Q}{4\pi\epsilon_0r}\]
\[\frac{dV}{dr}\hat{r}=-\frac{Q}{4\pi\epsilon_0r^2}\hat{r}\quad\Rightarrow\quad\vec{E}=-\frac{dV}{dr}\hat{r}\]
\[\left|E_x\right|=\left|{\frac{\Delta V}{\Delta x}}\right|_{yz},\quad\left|E_y\right|=\left|{\frac{\Delta V}{\Delta y}}\right|_{xz},\quad\left|E_z\right|=\left|{\frac{\Delta V}{\Delta z}}\right|_{xy}\]
\[\Rightarrow\vec{E}=-(\frac{\partial V}{\partial x}\hat{x}+\frac{\partial V}{\partial y}\hat{y}+\frac{\partial V}{\partial z}\hat{z})=-grad\ V\]

\end{document}