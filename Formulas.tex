% !TEX root = Formulas.tex  

\documentclass{article}
\usepackage[utf8]{inputenc}
\usepackage{empheq} % autoloads mathtols and amsmath

\title{Physics Formulas}
\author{Wei Meng Soh \thanks{Chong Hwa Independent High School Kuala Lumpur}}
\date{February 2021}

\begin{document}
\maketitle
\begin{abstract}
    This is a list of formulas for physics...
\end{abstract}

\section{Thermometry}
\subsection*{Type of thermometers}

\subsubsection*{Liquid thermometer}
Thermometric Property: \(\Delta V \propto \Delta \theta\)

\noindent Formulae:
\[\theta = \frac{\ell_\theta - \ell_0}{\ell_{100}-\ell_0} \times 100^{\circ} \mathrm{C} \quad, \quad T=\frac{\ell_T-\ell_{00}}{\ell_{tr}-\ell_{00}} \times 273.16\mathrm{~K} \]

\subsubsection*{Gas thermometer}
Thermometric Property: \(\Delta P \Delta V \propto \Delta \theta \quad (\mathrm{where} \ P = \rho gh)\)

\noindent Formulae:
\[\theta = \frac{P_\theta V_\theta - P_0V_0}{P_{100} V_{100} - P_0V_0} \times 100^{\circ} \mathrm{C} \quad, \quad T=\frac{P_T V_T}{P_{tr} V_{tr}} \times 273.16\mathrm{~K} \]

\subsubsection*{Resistance thermometer}
Thermometric Property: \(\Delta R \propto \Delta \theta \quad (\mathrm{where} \ (\mathrm{i}) R=\frac{P}{Q}\times S \ (\mathrm{ii})\ R_t=R_0(1+at+bt^2))\)

\noindent Formulae:
\[\theta = \frac{R_\theta - R_0}{R_{100}-R_0} \times 100^{\circ} \mathrm{C} \quad, \quad T=\frac{R_T}{R_{tr}} \times 273.16\mathrm{~K} \]


\subsubsection*{Thermoelectric thermometer}
Thermometric Property: \(\Delta \varepsilon \propto \Delta \theta\)

\noindent Formulae:
\[\theta = \frac{\varepsilon_\theta - \varepsilon_0}{\varepsilon_{100}-\varepsilon_0} \times 100^{\circ} \mathrm{C} \quad, \quad T=\frac{\varepsilon_T-\varepsilon_{00}}{\varepsilon_{tr}-\varepsilon_{00}} \times 273.16\mathrm{~K} \]

\section{Calorimetry }
\subsection*{Heat Capacity and specific heat capacity}

\subsubsection*{Heat Capacity}
\[C = \frac{Q}{\Delta T}\quad (\mathrm{JK^{-1}})\]

\subsubsection*{Specific Heat Capacity}
\[c = \frac{Q}{m\Delta T}\quad (\mathrm{Jkg^{-1}K^{-1}})\]

\subsubsection*{Molar Heat Capacity}
\[C_v = \frac{Q}{n\Delta T} \ (\mathrm{Jmol^{-1}K^{-1}})\quad , \quad C_p = \frac{Q}{n\Delta T} \ (\mathrm{Jmol^{-1}K^{-1}})\]

\subsection*{Measurement of specific heat capacity}

\subsubsection*{Method of Mixture}
\[mc(\theta_3-\theta_2)=m_wc_w(\theta_2-\theta_1)+m_cc_c(\theta_2-\theta_1)\]

\subsubsection*{Electrical Heating Method}
\[VIt=(mc_\ell+C)\Delta \theta\]

\subsubsection*{Continuous Flow Method (Callendar \& Barnes' method)}

\begin{empheq}[left=\empheqlbrace]{align}
    \ V_1I_1t = m_1c(\theta_2-\theta_1)+ht\\
    \ V_2I_2t = m_2c(\theta_2-\theta_1)+ht
\end{empheq}

\subsection*{Specific Latent Heat}
\[L_f = \frac{Q}{m}\quad (Jkg^{-1})\quad ,\quad L_v = \frac{Q}{m}\quad (Jkg^{-1})\]

\subsubsection*{Finding specific latent heat of fusion of ice}
\[m_1c_w(\theta_1-\theta_2)+C(\theta_1-\theta_2)=mL_f+mc_w(\theta_2-0)\]

\subsubsection*{Finding specific latent heat of vaporisation of water}
\[mL_v+mc_w(100-\theta_2)=(m_1c_w+C)(\theta_2-\theta_1)\]

\subsection*{Thermal Expansion of solid}

\subsubsection*{Linear Expansion}
\[\alpha=\frac{l_2-l_1}{(\theta_2-\theta_1)l_1}\quad \Rightarrow\quad l_2=l_1[1+\alpha(\theta_2-\theta_1)]\]

\subsubsection*{Area Expansion}
\[\beta=\frac{A_2-A_1}{(\theta_2-\theta_1)A_1}\quad \Rightarrow\quad A_2=A_1[1+\beta(\theta_2-\theta_1)]\]
\[\beta=2\alpha\]

\subsubsection*{Volume Expansion}
\[\gamma=\frac{V_2-V_1}{(\theta_2-\theta_1)V_1}\quad \Rightarrow\quad V_2=V_1[1+\gamma(\theta_2-\theta_1)]\]
\[\gamma=3\alpha\]

\subsection*{Thermal Expansion of Liquid}
\[\gamma_\ell=\frac{V_1-V_0}{V_0\Delta \theta}\quad \Rightarrow\quad V_1=V_0(1+\gamma_\ell \Delta \theta)\]
\[3\alpha_c=\gamma_c=\frac{V_1'-V_0}{V_0\Delta \theta}\quad \Rightarrow\quad V_1'=V_0(1+\gamma_c \Delta \theta)\]
\[\gamma_a=\frac{V_1-V_1'}{V_0\Delta \theta}\quad \Rightarrow\quad \gamma_\ell=\gamma_a+\gamma_c\]

\end{document}